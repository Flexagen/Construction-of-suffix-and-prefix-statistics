\documentclass[a4paper,12pt]{article}

\usepackage{ucs}
\usepackage[utf8x]{inputenc}
\usepackage[russian]{babel}
%\usepackage{cmlgc}
\usepackage{graphicx}
\usepackage{listings}
\usepackage{xcolor}
%\usepackage{courier}
\usepackage{wrapfig}
\usepackage{amsmath}

\makeatletter
\renewcommand\@biblabel[1]{#1.}
\makeatother

\newcommand{\myrule}[1]{\rule{#1}{0.4pt}}
\newcommand{\sign}[2][~]{{\small\myrule{#2}\\[-0.7em]\makebox[#2]{\it #1}}}

% Поля
\usepackage[top=20mm, left=30mm, right=10mm, bottom=20mm, nohead]{geometry}
\usepackage{indentfirst}

% Межстрочный интервал
\renewcommand{\baselinestretch}{1.50}

\usepackage{array}
\newcolumntype{P}[1]{>{\centering\arraybackslash}p{#1}}

\usepackage{hyperref}
\hypersetup{
    colorlinks,
    citecolor=black,
    filecolor=black,
    linkcolor=black,
    urlcolor=black
}

\begin{document}
\thispagestyle{empty}
\begin{center}
\renewcommand{\baselinestretch}{1}
{\normalsize {Министерство науки и высшего образования Российской Федерации\\
Федеральное государственное бюджетное образовательное\\
учреждение высшего образования\\}
\large
{\sc <<Петрозаводский государственный университет>>\\
Институт математики и информационных технологий
}
}
\end{center}


\begin{center}
09.03.04 - Программная инженерия\\
Профиль направления подготовки бакалавриата\\
<<Системное и прикладное программное обеспечение>>\\
\end{center}

\vfill

\begin{center}
\medskip
	{\Large \sc {Отчёт о прохождении производственной практики\\}}
\end{center}

\vfill
\vfill
\vfill

\medskip

\begin{flushright}
\parbox{9cm}{%
\renewcommand{\baselinestretch}{1.2}
\normalsize
	Выполнил:\\
студент 2 курса группы 22207
\begin{flushright}
Scamer Cum Cumych
\end{flushright}
Место прохождения практики: \\
Кафедра информатики и математического обеспечения\\

Сроки прохождения практики: \\
30.05-09.06\\

Руководитель практики:\\
к.т.н., доцент\\
Богоявленская Ольга Юрьевна\\

\begin{flushright}
Оценка
  \sign[]{4cm}
\end{flushright}

\begin{flushright}
Дата
  \sign[]{4cm}
\end{flushright}
}
\end{flushright}

\vfill

\begin{center}
\large
    Петрозаводск --- 2023
\end{center}

\newpage
\tableofcontents

\newpage
\section*{Введение}
\addcontentsline{toc}{section}{Введение}
В наши дни идея генерации текста по математическим моделям (Natural Language Generation) наибирает всё большую популярность и практические примеры использования в повседневной жизни. Современные решения, как правило, основываются на колоссальных нейронных сетях, включающих в себя триллионы нейронов. Но, если ограничиться генерацией текста по статистическим моделям, то данная задача изящно решается с помощью алгоритма на основе Марковских цепей [1].	\\

	\textbf{Цель практики - } реализовать алгоритм построения статистики суффиксов и префиксов по заданным текстам.\\

	\textbf{Задачи производственной практики:}
\begin{enumerate}
	\item {Ознакомление с теорией и литературой по генерации текстов с помощью Марсковских цепей;}
	\item {Создание собственной структуры данных для хранения префиксов текста на C++ и её последующая интеграция в Python;}
	\item {Создание программного модуля по подсчёту и анализу префиксов и суффиксов в тексте;}
	\item {Тестирование разработанного модуля и стуктуры данных.}
\end{enumerate}

	Организация и кооперация с другими разработчиками данной задачи (Кирилловым Иваном и Афанасьевым Артёмом), а также контроль версий программного кода и распредление подзадач осуществлялось с помощью \href{https://github.com/Flexagen/Construction-of-suffix-and-prefix-statistics}{GitHub}.


\newpage
\addcontentsline{toc}{section}{Список литературы}
\begin{thebibliography}{}
	\bibitem{} {Керниrан, Брайан У., Пайк, Роб. Практика программирования. : Пер. с англ. - М. : ООО "И.Д. Вильямс", -288 с.}
\end{thebibliography}
\end{document}